\documentclass[12pt]{article}

% Essential Packages
\usepackage{amsmath} % Advanced math typesetting
\usepackage{amsfonts} % Mathematical fonts
\usepackage{amssymb} % Mathematical symbols
\usepackage{geometry} % Page layout
\usepackage{graphicx} % Include graphics
\usepackage{hyperref} % Hyperlinks
\usepackage{natbib} % Bibliography
\usepackage{authblk} % Author affiliation

% Page Layout
\geometry{a4paper, margin=1in}
% Title
\title{Computationally Solving a Partial Differential Equation using a Single Logical Operator}

% Author
\author[1]{Gabriel Q. Tucker}
\affil[1]{The Ohio State University, Department of Philosophy}
\affil[2]{Columbus, OH, United States}

% Date
\date{\today}

\begin{document}

\maketitle

% Abstract
\begin{abstract}
The purpose of this paper is A) to serve as a concrete demonstration that you can solve a high-order mathematical proof using only binary logic; and B) to serve as a tutorial on how to both represent and compute high-order expressions in low-order languages.
\end{abstract}

% Keywords
\vspace{0.5cm}
\noindent \textbf{Keywords:} Second-Order Predicate Logic, Partial Differential Equations, Computational Mathematics

% Introduction
\section{Introduction}
To solve our PDE, we will use only the NAND operator—the Sheffer stroke ($\uparrow$):\\

\begin{tabular}{cc|c}
A & B & \({A \uparrow B} \) \\ \hline
T & T & F \\
T & F & T \\
F & T & T \\
F & F & T \\
\end{tabular}\\

Using the Sheffer stroke, we will build to the rules required to perform the following PDE problem.\\

The PDE we will solve is the Navier-Stokes Equation, used to describe the motion of viscous fluid substances,

\begin{equation}
0 = -\frac{dP}{dx} + \mu \frac{d^2u}{dx^2}
\end{equation}

where $P$ is the pressure, $\mu$ is the dynamic viscosity, and $u(x)$ is the velocity of the fluid as a function of position $x$. For simplicity, assume $P$ is a constant gradient, so $\frac{dP}{dx} = C$, where $C$ is a constant.

The equation simplifies to:

\begin{equation}
\mu \frac{d^2u}{dx^2} = C
\end{equation}

Integrating both sides with respect to $x$ gives:

\begin{equation}
\mu \frac{du}{dx} = Cx + K_1
\end{equation}

where $K_1$ is an integration constant. Integrating again with respect to $x$:

\begin{equation}
\mu u = \frac{C}{2}x^2 + K_1x + K_2
\end{equation}

where $K_2$ is another integration constant. Finally, solving for $u(x)$:

\begin{equation}
u(x) = \frac{C}{2\mu}x^2 + \frac{K_1}{\mu}x + \frac{K_2}{\mu}
\end{equation}

This expression for $u(x)$ represents the velocity profile of the fluid under the given assumptions and conditions. The constants $K_1$ and $K_2$ can be determined based on boundary conditions specific to a problem.

% Methodology
\section{Justification}
The language from which we will begin is second-order predicate logic.  This is because the ninth of the Peano axioms (as they are typically expressed) is a second-order axiom, since it involves induction over sets.\\

From first-order logic, we may assign terms, which are either quantifier-bound variables or constants, to predicates.  We are permitted to use the existential and universal quantifiers.\\

From second-order logic, 

% Results
\section{Results}
Present the findings of your study. Use figures, tables, and equations to support your results.

% Conclusion
\section{Conclusion}
Summarize the main findings and their implications. Discuss the limitations of your study and suggest future research directions.

% Acknowledgements (Optional)
\section*{Acknowledgements}
Acknowledge any support or contributions from individuals or organizations.

% References
\bibliographystyle{apalike}
\bibliography{references} % Assumes you have a bibliography file called references.bib

\end{document}